\chapter{Progress}

\section{Model Proposal: Context-Free Grammar (CFG) Explanations}

The research has evolved
from the study of Finite Automata (FA)
to more expressive computational models.
%
While FA provided a baseline for explaining
sequential behaviors,
%
many complex problems require the model to
``remember'' an arbitrary number of previous
inputs in the sequence
to determine the validity of subsequent inputs

Consider the abstract language $L=\{a^{n}b^{n} \mid n \geq 1\}$,
which represents a sequence where every `a' must
be matched by a corresponding `b'.

\begin{itemize}
  %
  \item The Limitation: A standard
  Finite Automaton (FA) possesses no
  stack-based memory.
  %
  Therefore, it cannot count the number
  of $a$'s to ensure they match the number
  of $b$'s once $n$ exceeds the number
  of states in the machine \cite{DragonBook}.
  \item The Explanation Failure:
  If a PDA were used to validate a sequence,
  it would process inputs.
  %
  Upon encountering a mismatch (e.g., $a^{4}b^{2}$),
  it might reject the sequence, but it lacks the
  the context to generate a contrastive
  explanation such as:
  ``The sequence is invalid because the third or fourth `a'
  was not closed by a matching `b'.''
  Instead, it can only report a
  ``transition failure'' at the specific index,
  obscuring the root cause of the error.

\end{itemize}

\begin{itemize}
  \item \textbf{From FA to PDA:}
  We propose the use of Pushdown Automata (PDAs)
  to model decision-making processes with memory.

  Unlike FA, the addition of a stack
  allows the description of more complex
  languages.
  Here the challenge is how to explain
  PDA decisions.
\end{itemize}

\section{Motivation: The Gap between Detection and Explanation}

One well established application of Context-Free Grammars
is in the design of programming languages and compilers.
Compilers are highly efficient at detecting when a
sequence of tokens fails to belong to a grammar.
%
However, a fundamental question is:
are they able of generating a useful \emph{explanation}
for why the failure occurred or how to fix it?

Consider the following C code, where a typo has
introduced a double opening bracket \texttt{\{\{} in the \texttt{for} loop:
%
\begin{lstlisting}[language=C]
int main(){
    for(int i=0; i<10; i++){{  // <--- Error: Double bracket
        printf("hello");
    }
}
\end{lstlisting}
%
When compiled (e.g., using GCC or Clang), the parser
consumes the input until it reaches
\textless\textless EOF\textgreater\textgreater (end-of-file),
finding only then that
EOF was not expected,
instead there is an incomplete structure, the token `\}'
is expected before \textless\textless EOF\textgreater\textgreater.
The resulting error message is:
%
\begin{verbatim}
error: expected '}' at end of input
   5 | }
     |  ^
\end{verbatim}
%
\noindent \textbf{The Explanation:}
While the compiler's output is correct,
the file ended while the stack still
contained an open brace that was not closed yet.
It is \textit{misleading}.
\begin{itemize}
  %
  \item \textbf{Root Cause:}
  The compiler points to line 5
  (the end of the file) as the location of the error.
  However, the root cause is located at line 2.
  %
  \item \textbf{Lack of Contrastive Reasoning:}
  A human (or a formal explainer) would identify
  that the input is ``almost correct''.
  %
  The explanation should not just report a
  missing symbol at the end
  but rather propose a \textit{minimal correction}.
  %
\end{itemize}
%
In this case, the minimal correction is not to add a brace at the end,
but to remove the redundant opening brace at the loop initialization:
%
\begin{lstlisting}[language=C]
int main(){
    for(int i=0; i<10; i++){
        printf("hello");
    }
}
\end{lstlisting}
%
This discrepancy motivates the need for our proposed formal framework.
We aim to move beyond isolated decisions to
explained decisions revealing
these minimal set of edits (Contrastive Explanations).

\section{Preliminaries} 

% \subsection{Pushdown Automata and Context-Free Grammars}

% To provide a foundation for the proposed explanation
% extraction methods, 
% standard definitions and notations
% for Pushdown Automata and Context-Free Grammar are adopted
% \cite{IntroAutomataTheory,HandbookTheoreticalContextFreeGrammar}.

% \begin{definition}[Pushdown Automaton]
%   A Pushdown Automaton (PDA) extends the capabilities of a
%   Finite Automaton by incorporating an infinite memory stack.
%   A PDA is formally defined as a 7-tuple
%   $\mathcal{A}=(Q,\Sigma,V,\delta,q^0,v^0,F)$, where:
%   \begin{itemize}
%     %
%     \item Q is a finite set of states.
%     %
%     \item $\Sigma$ is the input alphabet, a finite
%     terminal alphabet.
%     %
%     \item $V$ is the stack alphabet, a finite
%     nonterminal alphabet that can be pushed
%     onto or popped from the stack.
%     %
%     \item $\delta:Q\times(\Sigma\cup\{\epsilon\}) \times V \rightarrow \mathcal{P}(Q \times V^*)$
%     %
%     \footnote{$\mathcal{P}$ denotes the power set
%     (the set of all its subsets).
%     %
%     $V^*$ and $\Sigma^*$ denote the Kleene closures
%     of the stack and input alphabets,
%     respectively,
%     %
%     representing the sets of all finite strings formed
%     by those alphabets.}
%     %
%     is the transition function.
%     %
%     It dictates how the machine transitions between states and
%     modifies the stack based on the current state, input symbol,
%     and the top symbol of the stack.
%     %
%     \item $q^0 \in Q$ is the initial state.
%     \item $v^0 \in V$ is the initial pushdownstore symbol.
%     \item $F\subseteq Q$ is the set of accepting states.
%   \end{itemize}
% %
%   A configuration of $\mathcal{A}$ is a triple $c=(q, \gamma, x)$ in
%   $Q\times V^*\times\Sigma^*$. And the automaton moves from c into configuration
%   $c'=(q',\gamma', x')$, denoted as $c\vdash c'$ if:
%   %
%   \begin{itemize}
%     %
%     \item $\gamma = v \gamma_1(v\in V)$
%     %
%     \footnote{$\gamma_{1} \in V^*$ represents
%     the remaining symbols on the
%     stack below the top symbol $v$.
%     %
%     Thus, $\gamma$ represents the full current stack, formed by
%     the top symbol $v$ (to be processed) and the rest of the stack $\gamma_1$.}
%     %
%     , $x=ax' (a\in \Sigma)$,
%     $\gamma'= m \gamma_1(m\in V^*)$ and $(q', m)\in \delta(q, a, v)$, namely ``a-move'';
%     %
%     \item or $\gamma = v \gamma_1(v\in V)$, $x=x'$, $\gamma'= m \gamma_1(m\in V^*)$
%     and $(q',m)\in \delta(q,\epsilon,v)$, namely ``$\epsilon$-move''.
%   \end{itemize}
%   %
%   A \emph{word} is accepted by a pushdown automaton if, starting with an
%   empty stack, there is a path through the automaton such
%   that the automaton stops in an
%   accepting state after the entire string has been read.
%   %
%   The \emph{language} recognized by a PDA $\mathcal{A}$
%   is the set of all accepted words, denoted as $L(\mathcal{A})$.
%   %
% \end{definition}

% The following PDA recognizes the
% language of balanced parentheses
% (a subset of the Dyck language)
% %
% \footnote{
%   The Dyck language describes a set of strings
%   with balanced and properly
%   nested brackets (e.g., (), [], \{\})
%   %
%   \cite{HandbookTheoreticalContextFreeGrammar}.
%   The example focuses solely on non-empty
%   sequences of balanced ( and ).
% }
% %
% and is used throughout the document
% to illustrate the proposed ideas.
% %

% \begin{figure}[ht]
%   \centering
%     \begin{tikzpicture}[shorten >=1pt,node distance=3.5cm,on grid,auto]
    \node[state, initial] (q_0)   {$q_0$};
    \node[state] (q_1) [right=of q_0] {$q_1$};
    \node[state, accepting] (q_2) [right=of q_1] {$q_{f}$};

    \path[->]
    (q_0) edge node {$\epsilon, \epsilon \to B\,\$$} (q_1)
    (q_1) edge [loop above] node {
        \begin{tabular}{r c l}
        $\epsilon, B$ & $\to$ & \texttt{(} $B$ \texttt{)}\,$B$ \\
        $\epsilon, B$ & $\to$ & \texttt{(} $B$ \texttt{)} \\
        $\epsilon, B$ & $\to$ & \texttt{()} $B$ \\
        $\epsilon, B$ & $\to$ & \texttt{()} \\
        \texttt{(}, \texttt{(} & $\to$ & $\epsilon$ \\
        \texttt{)}, \texttt{)} & $\to$ & $\epsilon$
        \end{tabular}%
    } ()
          edge node {$\epsilon, \$ \to \epsilon$} (q_2);
\end{tikzpicture}
%     \caption{A Pushdown Automaton $\mathcal{A}$
%     accepting the language of balanced parentheses.
%     The symbol \$ is used as the bottom-of-stack marker.}
%   \label{fig:pda_example}
% \end{figure}
% \begin{example}

%   Let $\mathcal{A}$ be the PDA
%   that accepts the language generated by $G$ in 
%   \autoref{ex:grammar_balanced}.
%   The PDA uses a stack to track the depth of nesting,
%   pushing a symbol for every open parenthesis and popping
%   for every closed one. $B$ represents \textit{Balanced} in $G$.
  
%   \begin{itemize}
%       \item States: $Q = \{q_0, q_1, q_f\}$, $q^0=q_0$, $F=\{q_f\}$
%       \item Input Alphabet: $\Sigma = \{\texttt{(},\texttt{)}\}$
%       \item Stack Alphabet: $V = \{ \texttt{(}, \texttt{)}, B, \$ \}$, $v^0=B$
%       \item Transitions:
%       \begin{enumerate}
%           \item $\delta(q_0, \epsilon, \epsilon) = \{(q_1, B\,\$)\}$ \quad (Initialize stack with Start Symbol)
%           \item $\delta(q_1, \epsilon, B) = \{(q_1, \texttt{(}B\texttt{)}B),(q_1, \texttt{(}B\texttt{)}),(q_1, \texttt{()}B),(q_1, \texttt{()})\}$ \quad (Expand $B$)
%           \item $\delta(q_1, \texttt{(}, \texttt{(}\,) = \{(q_1, \epsilon)\}$ \quad (Match input `\texttt{(}' with stack `\texttt{(}')
%           \item $\delta(q_1, \texttt{)},\texttt{)}\,) = \{(q_1, \epsilon)\}$ \quad (Match input `\texttt{)}' with stack `\texttt{)}')
%           \item $\delta(q_1, \epsilon, \$) = \{(q_f, \epsilon)\}$ \quad (Accept if bottom marker is reached)
%       \end{enumerate}
%   \end{itemize}

%   \autoref{fig:pda_example} illustrates this PDA.
%     The configuration history of $\mathcal{A}$ for the input \texttt{()()} is:
% %
%   \begin{equation}
%     \begin{aligned}
%         & (q_0, \texttt{()()}, \epsilon) \\
%         \vdash \ & (q_1, \texttt{()()}, B\$) && \text{(Initialize)} \\
%         \vdash \ & (q_1, \texttt{()()}, \texttt{()}B\$) && \text{(Expand } B \to \texttt{()}B \text{)} \\
%         \vdash \ & (q_1, \texttt{)()}, \texttt{)}B\$) && \text{(Match `\texttt{(}')} \\
%         \vdash \ & (q_1, \texttt{()}, B\$) && \text{(Match `\texttt{)}')} \\
%         \vdash \ & (q_1, \texttt{()}, \texttt{()}\$) && \text{(Expand } B \to \texttt{()} \text{)} \\
%         \vdash \ & (q_1, \texttt{)}, \texttt{)}\$) && \text{(Match `\texttt{(}')} \\
%         \vdash \ & (q_1, \epsilon, \$) && \text{(Match `\texttt{)}')} \\
%         \vdash \ & (q_{f}, \epsilon, \epsilon) && \text{(Accept)}
%     \end{aligned}
%   \end{equation}
% \end{example}

\begin{definition}[Context-Free Grammar]
  A Context-Free Grammar (CFG)
  is defined as a 4-tuple $G=(V,\Sigma ,R,S)$, where:
  \begin{itemize}
    \item V (Variables/Non-terminals) is a finite set of variables (non-terminal symbols).
    \item $\Sigma$ (Terminals) is a finite set of terminal symbols, disjoint from V.
    \item R is a finite set of production rules of the
    form $A \rightarrow \alpha$, where $A\in V$ describes
    a variable and $\alpha\in(V \cup\Sigma)^*$ is a string
    of variables and terminals.
    \item $S\in V$ is the start variable.
  \end{itemize}
\end{definition}
%
A fundamental equivalence between CFGs and PDAs:
a language $L$ is context-free iff there exists a
PDA $\mathcal{A}$ such that $L(\mathcal{A})=L$
\cite{HandbookTheoreticalContextFreeGrammar, IntroAutomataTheory}.
%
This equivalence allows us to use grammar-based
parsing algorithms, such as CYK, to analyze the
behavior of PDAs.

The following PDA recognizes the
language of balanced parentheses
(a subset of the Dyck language)
%
\footnote{
  The Dyck language describes a set of strings
  with balanced and properly
  nested brackets (e.g., (), [], \{\})
  %
  \cite{HandbookTheoreticalContextFreeGrammar}.
  The example focuses solely on non-empty
  sequences of balanced ( and ).
}
%
and is used throughout the document
to illustrate the proposed ideas.
%

\begin{example}
  \label{ex:grammar_balanced}
  Let $G=(\{Balanced\}, \{\texttt{(},\texttt{)}\}, R, Balanced)$ be the grammar
  defined by the following 
  production rules $R$:
  %
  \begin{equation}
  \begin{aligned}
      Balanced &\to \texttt{(} \, Balanced \, \texttt{)} \, Balanced & \text{(Rule 1)}\\
      Balanced &\to \texttt{(} \, Balanced \, \texttt{)} & \text{(Rule 2)}\\
      Balanced &\to \texttt{(} \, \texttt{)} \, Balanced & \text{(Rule 3)}\\
      Balanced &\to \texttt{(} \, \texttt{)} & \text{(Rule 4)}
  \end{aligned}
  \end{equation}
  %
  This grammar generates the language of
  properly nested parentheses.
  For instance, the string \texttt{()()}
  can be derivated as follows.
  %
  \begin{equation}
  \begin{aligned}
      Balanced &\Rightarrow \texttt{( )} \, \mathbf{Balanced}  && \text{(Rule 3: parentheses and Balanced)} \\
               &\Rightarrow \texttt{( ) ( )} && \text{(Rule 4: Reduce `Balanced' to `()')}
  \end{aligned}
  \end{equation}
\end{example}

To analyze the behavior of the PDA
using grammar-based approaches, we must first standardize
the grammar structure (e.g. Chomsky Normal Form).
%
This enables the use of efficient
parsing algorithms 
like CYK (Cocke-Younger-Kasami) \cite{CYK1967, ParsingGuide}.

\subsection{Chomsky Normal Form}

Parsing algorithms often require the grammar to be in a
canonical form to ensure predictable execution complexity.

\begin{definition}[Chomsky Normal Form]
  A Context-Free Grammar $G=(V, \Sigma, R, S)$ is in
  \textit{Chomsky Normal Form} (CNF) \cite{IntroAutomataTheory} if every production rule
  in $R$ is of one of the following two forms:
  \begin{itemize}
      \item $A \rightarrow BC$ (where $A, B, C \in V$)
      \item $A \rightarrow a$ (where $a \in \Sigma$ and $a \neq \epsilon$)
  \end{itemize}
\end{definition}

For every CFG $G$ whose languge contains at least one string
other than $\epsilon$, then there is a grammar $G_1$
in Chomsky Normal Form, such that $L(G_1)=L(G)\setminus\{\epsilon\}$.

\begin{example}
  \label{ex:cnf_conversion}
  Consider the grammar $G$ from \autoref{ex:grammar_balanced}.
  We transform $G$ into an
  equivalent grammar $G'=(V', \Sigma, R', S)$ in CNF.
  
  \textbf{Step 1: Terminals to Non-terminals.}
  To introduce variables $L$ and $R$ for terminals `\texttt{(}' and `\texttt{)}'.
  \[ L \to \texttt{(} \quad \text{and} \quad R \to \texttt{)} \]
  %
  \textbf{Step 2: Binary decomposition.}
  To rewrite the original rules using new variables
  and break down productions into binary steps.
  
  The resulting production rules $R'$ are:
  \begin{center}
  \begin{tabular}{r r l l}
    Balanced  &$\to$&$ Nested\,Balanced$ & (Represents `\texttt{(} Balanced \texttt{)} Balanced' ) \\
              &$\mid$& $Unclosed\,R$ & (Represents `\texttt{(} Balanced \texttt{)}' ) \\
              &$\mid$& $Pair\,Balanced$ & (Represents `\texttt{( )} Balanced' ) \\
              &$\mid$& $L\,R$ & (Represents `\texttt{( )}' ) \\
    Nested &$\to$& $Unclosed\,R$ \\
    Unclosed &$\to$& $L\,Balanced$  \\
    Pair &$\to$& $L\,R$ \\
    L &$\to$& \texttt{(} \\
    R &$\to$& \texttt{)}
  \end{tabular}
  \end{center}
\end{example}

\subsection{The CYK Algorithm}

The Cocke-Younger-Kasami (CYK) \cite{CYK1967, ParsingGuide}
algorithm is a bottom-up
parsing method that given a CFG $G$ in CNF
determines whether a string $w$ belongs
to a language $L(G)$. It operates via
dynamic programming,
constructing a triangular table
where each cell $T[i,j]$ contains the set of
non-terminals that can generate
the substring of $w$ starting at $i$ and ending at $j$.

\begin{definition}[CYK Table Construction]
  \label{def:cyk}
  For an input string $w = w_1 w_2 \dots w_n$:
  \begin{enumerate}
      \item \textbf{Base Case:} For each $i \in \{1, \dots, n\}$,
      $T[i,i] = \{A \in V \mid A \to w_i\}$.
      \item \textbf{Recursive Step ($j > i$):}
      $T[i,j]$ contains $A$ if there exists
      a rule $A \to BC$ and a split point $k$ ($i \le k < j$)
      such that $B \in T[i,k]$ and $C \in T[k+1, j]$.
  \end{enumerate}
  The string is accepted if starting symbol $S \in T[1,n]$.
\end{definition}

\begin{example}
  The \autoref{tab:cyk_triangular} illustrates a 
  CYK Table construction to 
  verify the acceptance of the string $w = \texttt{()()}$
  using the CNF grammar derived in \autoref{ex:cnf_conversion}.
  %
  The top-right cell $T[1,4]$ represents the 
  entire string.
  %
  It contains the Start symbol $B$ (Balanced)
  because there is a rule `$Balanced \to Pair\,Balanced$'
  for a split point $k=2$ where $Pair\in T[1,2]$ and $Balanced\in T[3,4]$.
  
  Similarly, the cell $T[1,2]$ contains $P$ (Pair) because
  there is a rule $Pair \to L\, R$
  for a split point $k=1$ ($L\in T[1,1]$ and $R\in T[2,2]$).

  \begin{table}[h]
    \centering
    \renewcommand{\arraystretch}{1.5}
    \begin{tabular}{|c|c|c|c|}
    \hline
    $w_1=`('\,\{L\}$ & $\{B,P\}$ & $\emptyset$ & $\mathbf{\{B\}}$ \\ \cline{1-4}

    \multicolumn{1}{c|}{} & $w_2=`)'$ \{R\} & $\emptyset$ & $\emptyset$ \\ \cline{2-4}

    \multicolumn{1}{c}{} & \multicolumn{1}{c|}{} & $w_3=`('$ \{L\} & \{B,P\} \\ \cline{3-4}

    \multicolumn{1}{c}{} & \multicolumn{1}{c}{} & \multicolumn{1}{c|}{} & $w_4=`)'$ \{R\} \\ \cline{4-4}

    \end{tabular}
    \caption{CYK Table for $w=\text{()()}$. 
    The cells $T[i,j]$ correspond to the substring starting at $i$ and ending at $j$.
    The diagonal elements contain the input terminals and unary rules able to generate them.
    \textbf{Abbreviations:} $B$ = Balanced, $P$ = Pair}
    \label{tab:cyk_triangular}
  \end{table}
\end{example}

\section{Explaining Pushdown Automata decisions}

This report describes ongoing work to explain the
behaviour of a Pushdown Automaton (PDA) $\mathcal{A}$
on an input $w \in \Sigma^*$. Similar to explaining
Finite Automata decisions~\cite{explainingFA},
this problem can be viewed as a classifier mapping
input $w$ to a class in
${\cal K} = \{ \mbox{accept},\mbox{reject} \}$.
%
We propose two types of explanations:
\textbf{Abductive Explanations} (AXps), which answer
``Why does $\mathcal{A}$ accept/reject $w$?''; and
\textbf{Contrastive Explanations} (CXps), which answer
``How can $w$ be modified to alter the response of $\mathcal{A}$?''.

The class of explanations that are considered in this work 
are defined by regular expressions
%
formed by replacing
some characters in $w$ by $\Sigma$.

An \textbf{abductive explanation} relative to a word,
is a set of indices such that
if they are fixed, the prediction is going to remain
the same regardless of the other tokens.
%
A \textbf{contrastive explanation}
is a set of indices such that
if they were free, it would be possible to flip the
prediction.

Typically, accepted words in Context-Free Languages
are easy to modify to flip the prediction to \textbf{rejected}.
Conversely, rejected words are often robustly wrong,
requiring multiple coordinated changes (or single specific one)
to repair the word.

The following example illustrates this behaviour.

\begin{example}[Fragility of Acceptance]
Consider the accepted word $w = \texttt{()()}$.
The validity relies on every token.
Modifying any single index is sufficient
to flip the prediction to \textbf{rejected}.

Every single index constitutes a CXp:
\begin{itemize}
    \item \textbf{Index 1:} Changing `\texttt{(}' $\to$ `\texttt{)}' makes $\texttt{))()}$ (Rejected: starts with closing).
    \item \textbf{Index 2:} Changing `\texttt{)}' $\to$ `\texttt{(}' makes $\texttt{((()}$ (Rejected: unmatched open).
    \item \textbf{Index 3:} Changing `\texttt{(}' $\to$ `\texttt{)}' makes $\texttt{()))}$ (Rejected: unmatched close).
    \item \textbf{Index 4:} Changing `\texttt{)}' $\to$ `\texttt{(}' makes $\texttt{()((}$ (Rejected: unmatched open).
\end{itemize}

Since every feature is critical, the AXp
(the subset of features required to guarantee acceptance)
is the entire word.
\end{example}

The problem becomes significantly more
interesting for rejected words.

\begin{example}[Robustness of Rejection]
  \label{ex:rejected_parenthesis}
Consider the rejected word $w = \texttt{))))}$.
Changing a single index
is insufficient to flip the prediction to \textbf{accepted}.
%
There are only two Minimal CXps for this word:
%
\begin{itemize}
    \item $\text{CXp}_1 = \{1, 2\}$: Generates \texttt{\underline{((}))}.
    \item $\text{CXp}_2 = \{1, 3\}$: Generates \texttt{\underline{(})\underline{(})}.
\end{itemize}
%
From this, we can extract meaningful AXps that are sufficient
to \textit{guarantee} rejection:
%
\begin{itemize}
  %
  \item $\text{AXp}_1 = \{1\}$: The first symbol `\texttt{)}'
  guarantees rejection regardless of the remaining suffix.
  %
  \item $\text{AXp}_2 = \{2, 3\}$: The substring `\texttt{))}'
  at indices 2 and 3 guarantees rejection for any word of length 4.
  %
\end{itemize}
\end{example}

% Similarly to FXAI~\cite{delivering_trust,explainingFA}, we define a shorthand
% notation for explanations using subsets of positions in
% $w \in L(\mathcal{A})$:
% %
% \begin{align*}
%     %
%     \axp_l^u(S, w)= L(s_1 \cdots s_n) \mid s_i = w[i] \text{ if } i\in
%     S \text{ else } \Sigma \\
%     %
%     \cxp_l^u(S, w)= L(s_1 \cdots s_n) \mid s_i = w[i] \text{ if }
%     i\not\in S \text{ else } \Sigma
%     %
% \end{align*}

% \begin{example} \label{ex:indices}
%     %
%     Consider the setup of Example~\ref{ex:rejected_parenthesis}.
%     %
%     Given the word $w=\texttt{()()}$,
%     $L(\texttt{)}\Sigma\Sigma\Sigma) = \axp(\{1\}, w)$
%     and $L(\Sigma\texttt{))}\Sigma) = \axp(\{2,3\}, w)$
%     %
%     Similarly, $L(\Sigma\Sigma\texttt{))})=\cxp(\{1,2\}, w)$ and
%     $L(\Sigma\texttt{)}\Sigma\texttt{)})=\cxp(\{1,3\}, w)$.
%     %
%     \qed
%     %
% \end{example}

\subsection{Extracting One Formal Explanation}

\begin{definition}[CYK-based Verification of CXp]
  \label{def:cyk_cxp}
  For an input string $w = w_1 w_2 \dots w_n$ and a grammar in Chomsky Normal Form, the CYK algorithm is modified to verify if an index set $S \subseteq \{1, \dots, n\}$ is a CXp for a rejected word as follows:
  \begin{enumerate}
      \item \textbf{Base Case:} For each $i \in \{1, \dots, n\}$:
      \begin{itemize}
        %
        \item If $i \in S$: $T[i,i] = \{A \in V \mid \exists \alpha \in \Sigma, A \to \alpha \}$
        (all non-terminal symbols with at least one unary rule).
        %
        \item If $i \notin S$ (Fixed): $T[i,i] = \{A \in V \mid A \to w_i\}$ (as Standard CYK).
      \end{itemize}
      \item \textbf{Recursive Step ($j > i$):}
        Standard CYK update
      %\item \textbf{Recursive Step ($j > i$):} $T[i,j] = \{A \mid A \to BC, B \in T[i, k], C \in T[k+1, j] \text{ for some } i \le k < j\}$.
  \end{enumerate}
  The set $S$ is a valid CXp if the start symbol belongs to $T[1,n]$.
\end{definition}


\begin{table}[h]
  \centering
  \renewcommand{\arraystretch}{1.5}
  \begin{tabular}{|c|c|c|c|}
  \hline
  $w_1=\Sigma$ \{L,R\} & \{B,P\} & \{L,U\} & $\mathbf{\{B\}}$ \\ \cline{1-4}

  \multicolumn{1}{c|}{} & $w_2=\Sigma$ \{L,R\} & \{B,P\} & $\emptyset$ \\ \cline{2-4}

  \multicolumn{1}{c}{} & \multicolumn{1}{c|}{} & $w_3=\texttt{)}$ \{R\} & $\emptyset$ \\ \cline{3-4}

  \multicolumn{1}{c}{} & \multicolumn{1}{c}{} & \multicolumn{1}{c|}{} & $w_4=`)'$ \{R\} \\ \cline{4-4}

  \end{tabular}
  \caption{
    Modified CYK Table verifying $\texttt{))))}$.
    %
    Indices 1 and 2 are treated as wildcards ($\Sigma$).
    %
    \textbf{Abbreviations:}
    $B$ = Balanced, $P$ = Pair, $U$ = Unclosed
  }
  \label{tab:cyk_cxp}
\end{table}

\begin{algorithm}[tb]
  \caption{\textsc{ExtractCXp} -- a Single CXp Extraction}
\label{alg:extract_cxp}
\textbf{Input}: Context-Free Grammar $G$, Candidate set $\YYY$ (initially $\{1..\dots|w|\}$), word $w$ \\
\textbf{Output}: Minimal CXp $\YYY$
\begin{algorithmic}[1]
  \IF{\textbf{not} \textsc{IsCXp}$(G, \YYY, w)$}
      \STATE \textbf{return} $\bot$
  \ENDIF
  \FORALL{$i \in \YYY$} 
    \IF{\textsc{IsCXp}$(G, \YYY \setminus \{i\}, w)$}
      \STATE $\YYY \gets \YYY \setminus \{i\}$
    \ENDIF
  \ENDFOR
  \STATE \textbf{return} $\YYY$
\end{algorithmic}
\end{algorithm}



Algorithm~\ref{alg:extract_cxp} extracts a minimal CXp
using a greedy deletion strategy.
%
Initially, $\YYY$ includes all indices
(effectively treating the whole word as wildcards).
%
The algorithm iterates through the candidates,
attempting to ``recover'' the original token
$w_i$ at each position.
%
If the function \textsc{IsCXp} (Definition~\ref{def:cyk_cxp})
confirms that the word can still be corrected \textit{without}
changing index $i$,
then $i$ is redundant and removed from the explanation,
%
otherwise, $i$ is part of the explanation.
%
The result is a minimal CXp: removing any index from $\YYY$
would find a pattern containing no valid strings in $L(G)$.


% Given a rejected word w, Algorithm~\ref{alg:extract_cxp} extracts a
% minimal CXp by iteratively refining a candidate set Y.
% Initially, the algorithm starts with a full set of ``free'' indices
% represented by the language $L(\Sigma_1...\Sigma_{|w|})$, 
% effectively searching for any accepted word of length $|w|$
% by the Context-Free Grammar G.

% The algorithm attempts to ``recover''
% characters from the original word w by
% iterating through the indices in Y. For each index $i\in\YYY$:
% \begin{itemize}
% \item The index is temporarily removed from the candidate set,
% meaning the original character $w_i$ is restored at that position.
% %
% \item The function \textsc{IsCXp} (as described in Definition~\ref{def:cyk_cxp})
% verifies if an accepted word can still be generated given the remaining wildcards.
% %
% \item If \textsc{IsCXp} returns true,
% the character $w_i$ is considered redundant to the explanation
% of rejection and is permanently removed from Y.
% %
% \item Otherwise, $w_i$ is a necessary token of the minimal set
% required for a correction and is part of the resulting CXp.
% \end{itemize}

% The resulting output is a \textit{minimal} CXp.
% %
% It represents a smallest subset of indices such that,
% %
% if those specific characters are allowed to change to any terminal
% in $\Sigma$, the word can be ``fixed'' to satisfy the grammar.
% Any further reduction of the set $\YYY$ would result in a
% pattern that contains no valid strings in $L(G)$.

\textbf{Abductive Explanations} (AXps) are extracted
similarly.
%
Since AXps represent sufficient conditions for rejection,
a set $S$ is a valid AXp if treating indices in $S$ as
fixed (and all others as wildcards) results in the
start symbol not appearing in $T[1,n]$.
