\chapter{Formal Explainability}

\subsection{Classification Problems and Formal Explanations}

Following XAI literature for classification problems
\cite{delivering_trust,darwiche2023logic}, we consider
a classifier $\kappa:~\mathbb{F}\rightarrow \mathcal{K}$
over a feature space $\mathbb{F}=\prod_{i=1}^{m} \mathbb{D}_i$
and a set of classes K.

For an instance $\mathbf{v}\in\mathbb{F}$ with prediction
$\kappa(\mathbf{v})=~c\in~\mathcal{K}$, an
\emph{abductive explanation} 
$\mathcal{X}\subseteq \mathcal{F}$ \emph{sufficient} for the prediction.
%
Formally, $\mathcal{X}$ is defined as:
%
\begin{equation}
    %
    \label{eq:axp}
    %
    \forall (\textbf{x}\in \mathbb{F}). \left[\bigwedge\nolimits_{i
    \in \mathcal{X}} (x_i=v_i)\right]\rightarrow
    \left(\kappa(\mathbf{x})=c\right)
    %
\end{equation}
%
Similarly, a \emph{contrastive explanation} (CXp) is a
minimal subset of features $\mathcal{Y}\subseteq\mathcal{F}$
that, if allowed to change, enables the prediction's alteration.
%
Formally, a contrastive explanation $\mathcal{Y}$ is defined as follows:
%
\begin{equation}
    %
    \label{eq:cxp}
    %
    \exists (\textbf{x}\in \mathbb{F}). \left[\bigwedge\nolimits_{j
    \not\in \mathcal{Y}} (x_j=v_j)\right] \land
    \left(\kappa(\mathbf{x}) \neq c\right)
    %
\end{equation}
%
Observe that abductive explanations are used to explain \emph{why} a
prediction is made by the classifier $\kappa$ for a given instance, while
contrastive explanations can be seen to answer \emph{why not} another
prediction is made by $\kappa$.
%
Alternatively, CXps can be seen as answering \emph{how} the
predication can be changed.

Importantly, abductive and contrastive explanations are known to enjoy
a minimal hitting set duality relationship~\cite{axpcxp}.
%
Given $\kappa(\mathbf{v})=~c$, let $\mathbb{A}_\mathbf{v}$ be the
complete set of AXps and $\mathbb{C}_\mathbf{v}$ be the complete set
of CXps for this prediction.
%
Then each AXp $\mathcal{X}\in\mathbb{A}_\mathbf{v}$ is a minimal
hitting set of $\mathbb{C}_\mathbf{v}$ and, vice versa, each CXp
$\mathcal{Y}\in\mathbb{C}_\mathbf{v}$ is a minimal hitting set of
$\mathbb{A}_\mathbf{v}$.\footnote{Given a collection of sets
$\mathbb{S}$, a \emph{hitting set} of $\mathbb{S}$ is a set $H$ such
that for each $S \in \mathbb{S}$, $H \cap S \neq \emptyset$. A
hitting set is \emph{minimal} if no proper subset of it is a
hitting set.}
%
This fact is the basis for the algorithms used for formal explanation
\emph{enumeration}~\cite{delivering_trust,ffa}.