% content moved from root `content.tex`

\section{Project Summary \& Refinements}

\begin{frame}{Summary}
    \textbf{Thesis Title:} Formal Explainability
    for Artificial Intelligence in
    Dynamic Environments.
    
    \vspace{0.3cm}
    \textbf{Goal:} Deliver explanations for
    sequential decision-making models.
    \vspace{0.3cm}
    % To shift from
    % heuristic XAI to
    % \textit{formal, mathematically guaranteed} explanations for sequential decision-making models.

    \textbf{The Research Roadmap:}
    \begin{enumerate}
        \item \textbf{Finite Automata (Completed):}
        Explaining Finite Automata decisions
        and baseline sequential decisions.
        \item \textbf{Context-Free Grammars (Current):}
        Explaining languages Context-Free Languages
        requiring (stack-based memory).
        \item \textbf{Stochastic Models (Future):}
        %Extracting the ``most likely'' formal
        Explaining sequential decisions in
        probabilistic environments.
    \end{enumerate}
\end{frame}

\begin{frame}{Summary}

    \textbf{Goal:} Deliver explanations for
    sequential decision-making models.

    \begin{tikzpicture}[
        node distance=0.8cm,
        roadmapbox/.style={
            rectangle,
            rounded corners,
            draw,
            thick,
            text width=3.6cm, % Fits 3 boxes across a standard Beamer slide
            align=left,
            minimum height=3.5cm,
            inner sep=6pt,
            font=\small
        },
        completed/.style={roadmapbox, fill=green!10, draw=green!70!black},
        current/.style={roadmapbox, fill=blue!10, draw=blue!70!black},
        future/.style={roadmapbox, fill=gray!10, draw=gray!70!black},
        arrow/.style={-{Stealth[scale=1.2]}, line width=1.5pt, draw=gray!80}
    ]

    % --- Nodes ---
    \node[completed] (fa) {
        \textbf{1. Finite Automata}\\
        \vspace{0.1cm}
        \textit{\textcolor{green!70!black}{(Completed)}}\\[0.3cm]
        Explaining FA decisions and baseline sequential decisions.
    };

    \node[current, right=of fa] (cfg) {
        \textbf{2. Context-Free Grammars}\\
        \vspace{0.1cm}
        \textit{\textcolor{blue!70!black}{(Current)}}\\[0.3cm]
        Explaining Context-Free Languages requiring stack-based memory.
    };

    \node[future, right=of cfg] (stoch) {
        \textbf{3. Stochastic Model}\\
        \vspace{0.1cm}
        \textit{\textcolor{gray!70!black}{(Future)}}\\[0.3cm]
        Explaining sequential decisions in probabilistic environments.
    };

    % % --- Arrows ---
    \draw[arrow] (fa) -- (cfg);
    \draw[arrow] (cfg) -- (stoch);

    \end{tikzpicture}
\end{frame}

\section{Completed Work: Finite Automata}

\begin{frame}{Explaining Finite Automata}
    % \begin{itemize}
    %     \item We view a
    %     FA as a mapping from an input $w \in \Sigma^*$
    %     to a class $\mathcal{K} = \{\text{Accept}, \text{Reject}\}$.
    %     \item Explanations are
    %     Regular Expressions formed by substituting selected characters
    %     in $w$ with a wildcard token ($\Sigma$).
    % \end{itemize}

    \vspace{0.2cm}
    
    \begin{center}
    \scalebox{0.8}{
    \begin{tikzpicture}[shorten >=1pt, node distance=1cm, on grid, auto, initial text=, thick]
      \tikzstyle{every state}=[fill={rgb:black,1;white,10}, minimum size=18pt, inner sep=1pt]

      % Nodes (compressed vertically)
      \node[state, initial]            (q0) {0};
      \node[state, opacity=0.3]        (q1) [right=1.8cm of q0, yshift=0.4cm] {1};
      \node[state]                     (q2) [right=1.8cm of q0, yshift=-0.4cm] {2};
      \node[state, opacity=0.3]        (q3) [right=1.8cm of q1] {3};
      \node[state, opacity=0.3]        (q4) [right=1.8cm of q3] {4};
      \node[state, accepting, double]  (q5) [right=3.6cm of q2] {5};

      % Edges
      \path[->]
        (q0) edge[draw opacity=0.3] node[xshift=8pt, yshift=-1pt] {\texttt{a}} (q1)
             edge node[xshift=-5pt, yshift=-2pt] {\texttt{b}} (q2)
        (q1) edge[draw opacity=0.3] node[yshift=-2pt] {\texttt{a,b}} (q3)
        (q2) edge node[yshift=-2pt] {\texttt{b}} (q5)
             edge[draw opacity=0.3] node[xshift=2pt, yshift=-2pt] {\texttt{a}} (q3)
        (q3) edge[draw opacity=0.3] node[yshift=-2pt] {\texttt{a}} (q4)
             edge[draw opacity=0.3] node[xshift=-3pt, yshift=-2pt] {\texttt{b}} (q5)
        (q4) edge[loop right, draw opacity=0.3] node[xshift=-2pt] {\texttt{a,b}} (q4)
        (q5) edge[loop right] node[xshift=-2pt] {\texttt{a,b}} (q5);
    \end{tikzpicture}
    }
    \end{center}

    \vspace{-0.2cm} % Pulls the explanation boxes slightly closer to the automaton

    % Visualizing AXp vs CXp
    \begin{center}
    \scalebox{0.8}{ % Scales the token boxes to match the aesthetic
    \begin{tikzpicture}[
        node distance=0.2cm,
        token/.style={rectangle, draw, minimum size=0.55cm, font=\ttfamily},
        wildcard/.style={rectangle, draw, minimum size=0.55cm, font=\ttfamily},
        label/.style={font=\small\bfseries}
    ]
        % Original Word
        \node[label] (lbl_w) at (-2, 0) {Input $w$:};
        \node[token] (w1) at (0,0) {b};
        \node[token, right=of w1] (w2) {b};
        \node[token, right=of w2] (w3) {b};
        \node[token, right=of w3] (w4) {b};
        \node[token, right=of w4] (w5) {b};
        \node[right=0.3cm of w5, font=\small\color{green!60!black}] {$\rightarrow$ Accept};

        % Abductive Explanation 1
        \node[label] (lbl_axp) at (-2, -0.9) {AXp 1:};
        \node[token, fill=blue!10] (a1) at (0,-0.9) {b};
        \node[token, fill=blue!10, right=of a1] (a2) {b};
        \node[wildcard, right=of a2] (a3) {$\Sigma$};
        \node[wildcard, right=of a3] (a4) {$\Sigma$};
        \node[wildcard, right=of a4] (a5) {$\Sigma$};
        \node[right=0.3cm of a5, font=\small\color{blue!70!black}] {$\rightarrow$ Guarantees Accept $L(\texttt{bb} \Sigma \Sigma \Sigma) \subseteq L(\mathcal{A})$};

        % Abductive Explanation 2
        \node[label] (lbl_axp) at (-2, -1.8) {AXp 2:};
        \node[wildcard] (a1) at (0,-1.8) {$\Sigma$};
        \node[wildcard, right=of a1] (a2) {$\Sigma$};
        \node[token, fill=blue!10, right=of a2] (a3) {b};
        \node[wildcard, right=of a3] (a4) {$\Sigma$};
        \node[wildcard, right=of a4] (a5) {$\Sigma$};
        \node[right=0.3cm of a5, font=\small\color{blue!70!black}] {$\rightarrow$ Guarantees Accept $L(\Sigma \Sigma \texttt{b} \Sigma \Sigma) \subseteq L(\mathcal{A})$};

        % Contrastive Explanation (Freeing indices)
        \node[label] (lbl_cxp) at (-2, -2.7) {CXp:};
        \node[wildcard, fill=red!10] (c1) at (0,-2.7) {$\Sigma$};
        \node[token, right=of c1] (c2) {b};
        \node[wildcard, fill=red!10, right=of c2] (c3) {$\Sigma$};
        \node[token, right=of c3] (c4) {b};
        \node[token, right=of c4] (c5) {b};
        \node[right=0.3cm of c5, font=\small\color{red!70!black}] {$\rightarrow$ Enables Reject $L(\Sigma \texttt{b} \Sigma \texttt{bb}) \not\subseteq L(\mathcal{A})$};

    \end{tikzpicture}
    }
    \end{center}
\end{frame}


% --- SLIDE 3: Milestone & Output ---
\begin{frame}{Outcomes: Enumeration \& Milestone}
    \begin{itemize}
        \item \textbf{Algorithmic Contribution:} 
        \begin{itemize}
            \item Leveraged this duality to develop algorithms for the formal \textbf{enumeration} of explanations in Finite Automata.
            \item Successfully maps the abstract concepts of XAI onto rigorous formal language properties.
        \end{itemize}
        
        \vspace{0.6cm}
        
        \begin{block}{Phase 1 Milestone Achieved}
            \textbf{Status:} Completed.\\
            \textbf{Output:} The formal definitions, duality proofs, and enumeration algorithms have been compiled and submitted to \textbf{ICALP 2026}.
        \end{block}
        
        \vspace{0.4cm}
        \item \textit{Transitioning to Phase 2:} With the baseline for regular languages established, we now scale the complexity to languages requiring memory.
    \end{itemize}
\end{frame}

\begin{frame}{Project Refinements Since Confirmation}
    \textbf{Refining the Scope and Methodology:}
    \vspace{0.3cm}
    
    \begin{itemize}
        \item \textbf{Shift from Automata Execution to Grammar Membership:}
        \begin{itemize}
            \item \textit{Initial Concept:} Explaining the execution trace and stack states of Pushdown Automata (PDA).
            \item \textit{Refinement:} Framed the problem purely around \textbf{CFG membership} ($w \in L(G)$). This allows us to leverage efficient grammar-based parsing algorithms (like CYK) rather than tracking infinite PDA states.
        \end{itemize}
        
        \vspace{0.3cm}
        
        \item \textbf{Targeted Probabilistic Scope:}
        \begin{itemize}
            \item \textit{Initial Concept:} Broad application to generic stochastic models.
            \item \textit{Refinement:} Narrowed the final phase to focus specifically on Probabilistic Context-Free Grammars (PCFGs) and Markov Models, ensuring the mathematical foundation built in Phase 2 scales directly into Phase 3.
        \end{itemize}
    \end{itemize}
\end{frame}


\begin{frame}{The Minimal Hitting Set Duality}
    \begin{itemize}
        \item \textbf{The Core Relationship:} Abductive and contrastive explanations share a formal duality \cite{axpcxp}.
        \item Every AXp is a minimal hitting set of the complete set of CXps, and vice versa.
        \item To flip a prediction (CXp), you must free at least one token from every reason that guarantees the current prediction (AXp).
    \end{itemize}

    \vspace{0.4cm}

    % TikZ Diagram: Hitting Set Duality
    \begin{center}
    \begin{tikzpicture}[
        node distance=3cm,
        setnode/.style={circle, draw, thick, minimum size=1.5cm, align=center, font=\small\bfseries},
        arrow/.style={<->, >=Stealth, thick, draw=black!70}
    ]
        \node[setnode, fill=blue!10, draw=blue!80] (axp) {$\mathbb{A}_w$\\ (All AXps)};
        \node[setnode, fill=red!10, draw=red!80, right=of axp] (cxp) {$\mathbb{C}_w$\\ (All CXps)};
        
        \draw[arrow] (axp) -- node[above, font=\small\itshape] {Minimal Hitting Set} node[below, font=\small\itshape] {Duality} (cxp);
        
        % Little floating text blocks to explain what they do
        \node[below=0.2cm of axp, font=\footnotesize, text width=2cm, align=center] {Fixes indices to\\ lock prediction};
        \node[below=0.2cm of cxp, font=\footnotesize, text width=2cm, align=center] {Frees indices to\\ flip prediction};
    \end{tikzpicture}
    \end{center}
\end{frame} 