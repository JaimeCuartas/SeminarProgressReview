
\begin{frame}{The Minimal Hitting Set Duality}
    \begin{itemize}
        \item \textbf{The Core Relationship:} Abductive and contrastive explanations share a formal duality \cite{axpcxp}.
        \item Every AXp is a minimal hitting set of the complete set of CXps, and vice versa.
        \item To flip a prediction (CXp), you must free at least one token from every reason that guarantees the current prediction (AXp).
    \end{itemize}

    \vspace{0.4cm}

    % TikZ Diagram: Hitting Set Duality
    \begin{center}
    \begin{tikzpicture}[
        node distance=3cm,
        setnode/.style={circle, draw, thick, minimum size=1.5cm, align=center, font=\small\bfseries},
        arrow/.style={<->, >=Stealth, thick, draw=black!70}
    ]
        \node[setnode, fill=blue!10, draw=blue!80] (axp) {$\mathbb{A}_w$\\ (All AXps)};
        \node[setnode, fill=red!10, draw=red!80, right=of axp] (cxp) {$\mathbb{C}_w$\\ (All CXps)};
        
        \draw[arrow] (axp) -- node[above, font=\small\itshape] {Minimal Hitting Set} node[below, font=\small\itshape] {Duality} (cxp);
        
        % Little floating text blocks to explain what they do
        \node[below=0.2cm of axp, font=\footnotesize, text width=2cm, align=center] {Fixes indices to\\ lock prediction};
        \node[below=0.2cm of cxp, font=\footnotesize, text width=2cm, align=center] {Frees indices to\\ flip prediction};
    \end{tikzpicture}
    \end{center}
\end{frame} 

\begin{frame}{Outcomes: Enumeration \& Milestone}
    \begin{itemize}
        \item \textbf{Algorithmic Contribution:} 
        \begin{itemize}
            \item Leveraged this duality to develop algorithms for the formal \textbf{enumeration} of explanations in Finite Automata.
            \item Successfully maps the abstract concepts of XAI onto rigorous formal language properties.
        \end{itemize}
        
        \vspace{0.6cm}
        
        \begin{block}{Phase 1 Milestone Achieved}
            \textbf{Status:} Completed.\\
            \textbf{Output:} The formal definitions, duality proofs, and enumeration algorithms have been compiled and submitted to \textbf{ICALP 2026}.
        \end{block}
        
        \vspace{0.4cm}
        \item \textit{Transitioning to Phase 2:} With the baseline for regular languages established, we now scale the complexity to languages requiring memory.
    \end{itemize}
\end{frame}


% \begin{frame}{Project Refinements Since Confirmation}
%     \textbf{Refining the Scope and Methodology:}
%     \vspace{0.3cm}
    
%     \begin{itemize}
%         \item \textbf{Shift from Automata Execution to Grammar Membership:}
%         \begin{itemize}
%             \item \textit{Initial Concept:} Explaining the execution trace and stack states of Pushdown Automata (PDA).
%             \item \textit{Refinement:} Framed the problem purely around \textbf{CFG membership} ($w \in L(G)$). This allows us to leverage efficient grammar-based parsing algorithms (like CYK) rather than tracking infinite PDA states.
%         \end{itemize}
        
%         \vspace{0.3cm}
        
%         \item \textbf{Targeted Probabilistic Scope:}
%         \begin{itemize}
%             \item \textit{Initial Concept:} Broad application to generic stochastic models.
%             \item \textit{Refinement:} Narrowed the final phase to focus specifically on Probabilistic Context-Free Grammars (PCFGs) and Markov Models, ensuring the mathematical foundation built in Phase 2 scales directly into Phase 3.
%         \end{itemize}
%     \end{itemize}
% \end{frame}